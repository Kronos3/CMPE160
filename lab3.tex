% A skeleton file for producing Computer Engineering reports
% https://kgcoe-git.rit.edu/jgm6496/KGCOEReport_template

\documentclass[CMPE]{KGCOEReport}

% The following should be changed to represent your personal information
\newcommand{\classCode}{CMPE 160}  % 4 char code with number
\newcommand{\name}{Andrei Tumbar}
\newcommand{\LabSectionNum}{4}
\newcommand{\LabInstructor}{Mr.\ Byers}	% The slash is to tell LaTeX that the period is between words
												% not sentences so it spaces correctly. It won't appear in the
												% final pdf
\newcommand{\TAs}{Sam Myers \\ Kobe Balin \\ Georgi Thomas}
\newcommand{\LectureSectionNum}{2}
\newcommand{\LectureInstructor}{Mr.\ Cliver}
\newcommand{\exerciseNumber}{1}
\newcommand{\exerciseDescription}{Electrical and Logical Characteristics of Gates}
\newcommand{\dateDone}{January 30th}
\newcommand{\dateSubmitted}{February 6th}

\usepackage{circuitikz}
\usepackage{tikz}

\def\coord(#1){coordinate(#1)}
\def\coord(#1){node[circle, red, draw, inner sep=1pt,pin={[red, overlay, inner sep=0.5pt, font=\tiny, pin distance=0.1cm, pin edge={red, overlay,}]45:#1}](#1){}}
\def\coordd(#1){node[circle, red, draw, inner sep=1pt,pin={[red, overlay, inner sep=0.5pt, font=\tiny, pin distance=0.1cm, pin edge={red, overlay,}]-45:#1}](#1){}}

\begin{document}
\maketitle

\section*{Abstract}
In the laboratory exercise, the logical characteristics of basic gates and also measurments regarding the delay of a chain of inverters was studied. In part one, switches from the input of the previous lab exercise were connected to three different ICs for testing. These two switches acted as inputs for the gates on the ICs. Measurements were taken for a two input NAND (74LS00), OR (74LS32), and an XOR (74LS86) gate. The purpose of this was to give students experience in using an IC's circuit diagram and wiring the IC to an LED. \par 
In the second part of this lab exercise, the time delay associated with a series of inverter gates was investigated. Five inverters were wired in a series and the delay after each inverter was probed and recorded. Measurements for each inverter delay was taken along with the period of oscillation.


\section*{Design Methodology}

\subsection*{Part One}
In part one of this lab, three different IC's werer tested using two switches from the previous exercise as input and one of the LEDs as output. \par
NAND, OR, and XOR IC were used. Because input, output, \(V_c\), and ground were located on the same relative pins on all three ICs, each IC could be swapped for the next type.

\begin{figure}[h]
\centering
\begin{circuitikz}[american, ]
	\ctikzset{multipoles/thickness=4}
	\ctikzset{multipoles/external pins thickness=2}	
	
	\draw (0,0) node[dipchip,
		num pins=14,
		external pins width=0.3,
		external pad fraction=4 ](C){IC};
	
	\draw (0,4) node[vcc](VCC){\(V_{CC} = 5V\)};
   	\draw (C.pin 14) -- ++(0.2,0) -- ++(0,0.5) -- ++(-1.45,0)  -- (VCC){};
   	\draw (C.pin 7) -- ++ (-4,0) node[ground, rotate=-90](gnd){};
   	\draw (C.pin 2) to[short, -*] ++(-2.94,0) -- (-4.2,3.5) to[R, bipoles/length=.8cm] ++(2,0) -- (0,3.5)  to[short, -*] ++ (0,0) (-4.2,1.2) to[ospst] ++(0,-2.87) to[short, -*] ++(0,0);
   	
   	\draw (C.pin 1) to[short, -*] ++(-2.44,0) -- ++(0,1.2) to[R, bipoles/length=.8cm] ++(2,0) -- (0,2.88)  to[short, -*] ++ (0,0) (-3.7, 1.8) -- (-3.7,1.2) to[ospst] ++(0,-2.87) to[short, -*] ++(0,0);
   	
   	\draw (C.pin 3) -- ++(-1,0) node[label={[font=\footnotesize]below:To LED}]{} to[short, -*] ++(0,0);

\end{circuitikz}

\caption{Example wiring for IC.}
\end{figure}

Figure 1 depicts switches to represent the input circuit from the last labs exercise. A resistor is placed between the power supply and the switch so that when the switch is closed, there is no short circuit between the ground and voltage supply. When the switch is open, 5V is supplied to the IC therefore inputing an ON signal to the device. When the switch is closed however, the wire is connected to ground therefore putting the wire at 0V.
A record of the output voltage (PIN 3) from each IC on every input combination as well as the results on the output LED were taken.

\subsection*{Part Two}
In part two, delay and oscillation period associated with inverters were investigated. Five inverters were wired in a series with a space for a probe to record the voltage on each node. The order that the inverters were connected proved unimportant as long as the final connection looped back to the first inverter.

\begin{figure}[h]
\centering
\begin{circuitikz}[american, ]
	\ctikzset{multipoles/thickness=4}
	\ctikzset{multipoles/external pins thickness=2}	
	
	\draw (0,0) node[american not port](i1){};
	\draw (2,0) node[american not port](i2){};
	\draw (4,0) node[american not port](i3){};
	\draw (6,0) node[american not port](i4){};
	\draw (8,0) node[american not port](i5){};
	
	\draw (i1.out) -- (i2.in);
	\draw (i2.out) -- (i3.in);
	\draw (i3.out) -- (i4.in);
	\draw (i4.out) -- (i5.in);
	\draw (i5.out) -- ++(1, 0) -- ++(0, -1.5) -- (-1.8, -1.5) -- ++(0, 1.5) -- (i1.in);

\end{circuitikz}

\caption{Circuit diagram for the inverter oscillator.}
\end{figure}

Figure 2 shows the circuit diagram for

\section*{Results and Analysis}

\section*{Conclusion}

\end{document}
