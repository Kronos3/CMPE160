\documentclass[twoside,11pt]{article}
%%%%% PACKAGES %%%%%%
\usepackage[english]{babel}
\usepackage{color}
\usepackage[margin=1in]{geometry}
\usepackage{minted}
\usepackage{tikz}
\usetikzlibrary{calc}
\usetikzlibrary{shapes,arrows,calc,arrows.meta}
\usepackage{circuitikz}
\usetikzlibrary{circuits.logic.IEC,calc}

\newcommand\assignmentNumber{10}
\newcommand\studentName{Andrei Tumbar}

\begin{document}

\title{\vspace{-3cm}Homework \assignmentNumber\\\studentName\vspace{-2cm}}
\maketitle
\vspace{-5cm}

\section{Write VHDL code (dataflow) to describe the circuit in two ways.}
\label{sec:background}


\begin{minted}{vhdl}
library IEEE;use IEEE.STD_LOGIC_1164.ALL;entity hw11_p1 is    Port ( signal A, _______________, C : in  STD_LOGIC;             signal _______________, _______________, _______________, in4, in3, in2, in1, in0  : in  STD_LOGIC;             signal mux_out : out STD_LOGIC);end hw11_p1; architecture behavioral of hw11_p1 is  begin  process (A, B, C, in7, _______________, _______________, _______________, in3, in2, in1, in0) is begin    case std_logic_vector'(A,B,C) is  when "000" => mux_out <= in0;  when "001" => mux_out <= _______________;  when "010" => mux_out <= _______________;  when "011" => mux_out <= in3;  when "100" => mux_out <= _______________;  when "101" => mux_out <= in5;  when "110" => mux_out <= _______________;  when others => mux_out <= in7; -- "111"end case;  end process;end behavioral

\end{minted}
\pagebreak

\section{Design a multiplexor circuit to describe the VHDL code.}

\begin{figure}[h!]
	\begin{center}
		\begin{tikzpicture}
		\draw (0,0)coordinate (O)--++(30:1)coordinate (A)--++(90:2)coordinate (B)--++(150:1)coordinate (C)--cycle;
		\draw ($(A)!0.5!(B)$)--++(0:1)node[right]{$Y$};
		\draw ($(O)!0.33!(A)$)--++(-90:1)--++(180:2)node[left]{$S_1$};
		\draw ($(O)!0.66!(A)$)--++(-90:1.75)--++(180:2.25)node[left]{$S_0$};
		\foreach \y/\t in {0.0/0,0.1/1,0.2/2,0.3/3} {
		\draw ($(C)! \y*2.2 + 0.2 !(O)$)--++(180:1) node[left] {$I \t$};}
		\end{tikzpicture}
	\end{center}
\end{figure}

\section{Design a 3:8 decoder circuit to describe the VHDL code.}

\begin{figure}[h!]
	\begin{center}
		\begin{circuitikz}[circuit logic IEC]
		\node[and gate,inputs={nnn},and gate IEC symbol={Decoder 3:8},text height=6cm,text width=3cm,
		 ] (A) {};

		\foreach \V/\X in {1/A,2/B,3/C} 
		{
		  \draw  ([xshift=-10pt]A.input \V) node[left] {$\X$} -- (A.input  \V);
		}

		\foreach \C/\B/\D in {0.111/111/7,.222/110/6,.333/101/5,.444/100/4,.555/011/3,.666/010/2,.777/001/1,.888/000/0} 
		{
		  \draw ( $ (A.south east)!\C!(A.north east) $ ) -- ++(10pt,0) node[left,xshift=-10] {$\B$} -- ++(20pt,0) node[left,xshift=20] {$Y_\D$};
		}

		\end{circuitikz}
	\end{center}
\end{figure}

\end{document}
